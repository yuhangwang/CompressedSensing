\section{Szczegółowy opis zadania}
\subsection{Specyfikacja projektu}
Celem projektu było zaproponowanie realistycznego modelu jednopikselowej kamery oraz przygotowanie co najmniej jednej metody odtwarzania rzadko spróbkowanego sygnału. Zadanie to wydaje się mieć wiele potencjalnych zastosowań, szczególnie w dziedzinie medycyny lub telekomunikacji. Oczekiwanym celem projektu jest maksymalne przyspieszenie działania algorytmu, poprzez wykorzystanie zasobów karty graficznej. 
\subsection{Szczegółowy opis algorytmu odtwarzania obrazu TVQC}
Algorytm TVQC został dokładnie zdefiniowany w dokumentacji pakietu $\ell_1$-MAGIC ~\cite{L1MagicNotes}. Metoda optymalizacji wykorzystana do odtworzenia zdjęcia została z kolei opisana w \cite{Boyd2004}. Poza przepisem matematycznym, przedstawiono również funkcjonowanie algorytmu na schemacie blokowym.
\subsubsection{Zdefiniowanie problemu}
Algorytm TVQC należy do grupy kwadratowych problemów optymalizacji (ang. \textbf{SOCP} - \textit{Second-order Cone Programming}). Przepiszmy więc problem ~\ref{eq:TV} do ich generycznej postaci:

\begin{equation}
\begin{split}
\min_{x,t} \sum_{ij} t_{i,j} \text{~~spełniający~~} &\|D_{ij}x \|_2 < t_{ij}, i,j = 1,...,n \\
&\|Ax - b \|_2 \le \epsilon.
\end{split}
\label{eq:SOCP-generic}
\end{equation}
Przyjmując
\begin{equation}
\label{eq:ftij}
f_{t_{ij}} = \frac{1}{2}\left(\|D_{ij}\|_2^2 - t^2_{ij}\right)\quad i,j=1,\ldots,n
\end{equation}
oraz
\[
f_\epsilon = \frac{1}{2}\left( \|Ax-b\|^2_2 - \epsilon^2\right),
\].
Mając
\[
\grad f_\epsilon = \bpm A^Tr \\ \vzero \epm,\quad 
\grad f_\epsilon \grad f_\epsilon^T =
\bpm  A^Trr^TA & \vzero \\ \vzero & \vzero \epm, \quad
\grad^2 f_\epsilon = \bpm A^*A & \vzero \\
\vzero & \vzero \epm
\]
gdzie $r = Ax - b$.
Także, 
\[
\grad^2 f_{t_{ij}} = \left(\begin{array}{cc} D_{ij}^*D_{ij} & \vzero \\
\vzero & -\delta_{ij}\delta_{ij}^T \end{array}\right)
\quad
\grad^2 f_\epsilon = \left(\begin{array}{cc} A^*A & \vzero \\
\vzero & \vzero \end{array}\right).
\]

Rozważana forma kwadratowa przedstawia się następująco:
\[ 
\bpm H_{11} & B\Sigma_{12} \\
\Sigma_{12}B^T & \Sigma_{22} \epm
\bpm \dx \\ \dt \epm =
\bpm D_h^T F_t^{-1}D_h x + D_v^TF_t^{-1}D_v x + f_\epsilon^{-1}A^Tr \\
-\tau\mathbf{1} - tf^{-1}_t \epm :=
\bpm w_1 \\ w_2 \epm.
\]
gdzie $(tf^{-1}_t)_{ij} = t_{ij}/f_{t_{ij}}$, oraz 
\begin{align*}
H_{11} = ~& D_h^T(-F_t^{-1})D_h ~+~ D_v^T(-F_t^{-1})D_v ~+~ B F_t^{-2}B^T ~- \\
 & f_\epsilon^{-1} A^TA ~+~ f_\epsilon^{-2} A^Trr^TA, \\
\Sigma_{12} = ~& -TF_t^{-2}, \\
\Sigma_{22} = ~& F_t^{-1} + F_t^{-2}T^2,
\end{align*}
Rugując $\dt$
\[
\dt = \Sigma_{22}^{-1}(w_2 - \Sigma_{12}\Sigma_{\partial h}D_h\dx - \Sigma_{12}\Sigma_{\partial v}D_v\dx),
\]
kluczową do rozwiązania formą kwadratową jest
\[
H'_{11}\dx  = 
w_1 - (D_h^T\Sigma_{\partial h} + D_v^T\Sigma_{\partial v} )\Sigma_{12}\Sigma_{22}^{-1}w_2
\]
gdzie
\begin{align*}
H'_{11} = ~& H_{11} - B\Sigma^2_{12}\Sigma_{22}^{-1}B^T \\
= ~& D_h^T(\Sigma_b\Sigma^2_{\partial h}-F_t^{-1})D_h ~+~
D_v^T(\Sigma_b\Sigma^2_{\partial v}-F_t^{-1})D_v ~+~\\
~& D_h^T(\Sigma_b\Sigma_{\partial h}\Sigma_{\partial v})D_v ~+~
D_v^T(\Sigma_b\Sigma_{\partial h}\Sigma_{\partial v})D_h ~-~ \\
~& f_\epsilon^{-1} A^TA ~+~ f_\epsilon^{-2} A^Trr^TA, \\
\Sigma_b = ~& F_t^{-2} - \Sigma_{12}^2\Sigma_{22}^{-1}.
\end{align*}

Powyższy układ równań jest dodatnio określony, co pozwala rozwiązać go metodą gradientu sprzężonego.

W celu rozwiązania powyższego problemu użyto metody \textit{log-barrier} \cite{Boyd2004}. Przekształca ona \textit{SOCP} w szereg problemów z ograniczeniami liniowymi:
\begin{equation}
\label{eq:logbarrier} 
\min_z ~ \<c_0,z\> + \frac{1}{\tau^k} \sum_{i}
-\log(-f_i(z)) \quad\text{spełniający}\quad A_0 z=b,
\end{equation}
gdzie $\tau^k > \tau^{k-1}$. Ograniczenia nierównościowe zostały włączone do funkcjonału w postaci \textit{funkcji kary} \footnote{Wybór $-\log(-x)$ dla funkcji bariery nie jest przypadkowy, gdyż posiada własność (samozgodność), która jest bardzo ważna w szybkiej zbieżności \eqref{eq:logbarrier} do \eqref{eq:SOCP-generic} zarówno w teorii, jak i praktyce.}, która jest nieskończona gdy ograniczenie jest przekroczone i gładka na pozostałym obszarze. Gdy $\tau^k$ rośnie, rozwiązanie $z^k$ problemu \eqref{eq:logbarrier} zmierza do rozwiązania $z^\star$  \eqref{eq:SOCP-generic}: 
Można wykazać, że $\<c_0,z^k\> - \<c_0,z^\star\> < m/\tau^k$, tzn. znajdujemy się o mniej niż $m/\tau^k$ od optymalnej wartości w $k$-tej iteracji ($m/\tau^k$ jest zwany {\em odstęp dualności}).
Ideą powyższej metody jest to, że każdy z gładkich podproblemów może być rozwiązany z dużą dokładnością przy jedynie kilku iteracjach metody Newtona, szczególnie, że możemy wykorzystać rozwiązanie $z^k$ jako punkt startowy podproblemu $k+1$.

W $k$-tej iteracji metody \textit{log-barrier}, metoda Newtona tworzy szereg kwadratowych aproksymacji równania \eqref{eq:logbarrier} i minimalizuje je rozwiązując układ liniowy. Aproksymacja kwadratowa funkcjonału
\[
f_0(z) = \<c_0,z\> + \frac{1}{\tau}\sum_i -\log(-f_i(z))
\]
w \eqref{eq:logbarrier} wokół punktu $z$ jest dana jako:
\[
f_0(z+\dz) ~\approx~ z + \<g_z,\dz\> + \frac{1}{2}\<H_z \dz,\dz\> ~:=~ q(z+\dz),
\]
gdzie $g_z$ jest gradientem
\[
g_z = c_0 + \frac{1}{\tau}\sum_i \frac{1}{-f_i(z)}\grad f_i(z)
\]
a $H_z$ jest macierzą Hessego
\[
H_z ~=~ \frac{1}{\tau}\sum_i \frac{1}{f_i(z)^2} \grad f_i(z) (\grad f_i(z))^T ~+ ~
\frac{1}{\tau}\sum_i \frac{1}{-f_i(z)}\grad^2 f_i(z).
\]
Przyjmując, że $z$ jest rozwiązaniem dopuszczalnym (tzn. $A_0 z=b$),  $\dz$ które minimalizuje $q(z+\dz)$ spełniające $A_0 z=b$ jest rozwiązaniem układu równań liniowych:
\begin{equation}
\label{eq:lbnewton}
\tau \bpm H_z & A_0^T \\ A_0 & \vzero \epm \bpm \dz \\ v \epm= -\tau g_z.
\end{equation}
(Wektor $v$, który może być interpretowany jako mnożniki Lagrange'a ograniczeń jakościowych w kwadratowym problemie optymalizacji, nie jest bezpośrednio wykorzystywany.)

Ponieważ w postawionym wyżej nie problemie nie występują ograniczenia równościowe, zachodzi $A_0=\vzero$. 

Mając $\dz$, znamy kierunek kroku metody Newtona. Długość kroku $s\leq 1$ jest tak wybrana, by
\begin{enumerate}
%
\item $f_i(z+s\dz) < 0$ dla każdego $i=1,\ldots,m$,
%
\item Funkcjonał dostatecznie zmalał:
\[
f_0(z+s\dz) < f_0(z) + \alpha s\dz \<g_z,\dz\>,
\]
gdzie $\alpha$ jest arbitralnym parametrem (przyjęto $\alpha=0.01$).  To wymaganie implikuje, że rozwiązanie musi być bliskie wyznaczonemu przez model liniowy w $z$.
%
\end{enumerate}
Ustawiamy $s=1$ i zmniejszamy o wielokrotność $\beta$ aż oba z tych warunków zostaną spełnione (przyjęto $\beta = 1/2$).


Ostatecznie, implementacja metody log-barrier przedstawia się w następującej postaci algorytmicznej:
\begin{enumerate}
\item Wejścia: rozwiązanie dopuszczalne $z^0$, tolerancja $\eta$,  parametry $\mu$ i początkowe $\tau^1$.  Ustaw $k=1$.
\item Rozwiąż \eqref{eq:logbarrier} metodą Newtona, wykorzystując $z^{k-1}$ jako punkt startowy. Rozwiązanie nazwij $z^k$.
\item Jeżeli $m/\tau^k < \eta$, zakończ i zwróć $z^k$.
\item W przeciwnym wypadku, ustaw $\tau^{k+1} = \mu\tau^k,~k=k+1$ i idź do kroku 2.
\end{enumerate}
W praktyce możemy z góry przewidzieć jak wiele iteracji będzie potrzebnych:
\[
\mathrm{liczba~iteracji} = \left\lceil \frac{\log m - \log\eta -\log\tau^1}{\log\mu}\right\rceil.
\]
Ostatnim problemem jest wybranie $\tau^1$. Dostało ono ustawione tak, by $m/\tau^1$ po pierwszej iteracji było równe $\<c_0,z^0\>$.