\section{Wstęp}
\subsection{Wprowadzenie}
Kierunek badań odtwarzania sygnału rzadkiego (ang. \textbf{CS} – \textit{Compressed Sensing}) jest stosunkowo nową i bardzo ciekawą gałęzią badań w~dziedzinie przetwarzania sygnałów cyfrowych (ang. \textbf{DSP} – \textit{Digital Signal Processing}). Próbuje on znaleźć odpowiedź na pytanie: jak najrzadziej mierzyć sygnał wejściowy, by móc go skutecznie odtworzyć? Takie podejście różni się od głównego nurtu DSP, próbujkącego sygnał powyżej częstotliwości Nyquista, co da szybkozmiennych sygnałów jest bardzo kosztowne. CS pozwala znacząco ograniczyć liczbę wykonanych pomiarów, jest więc szalenie korzystny  w~sytuacjach, w~których wykonanie pomiarów jest drogie (np. kamery pracujące w niewidzialnych pasmach częstotliwościowych), uciążliwe lub niebezpieczne (aparatura medyczna – MRI, Tomografia komputerowa)[Źródło] lub gdy trudno zapewnić transmisję o~dużej przepustowości. 

Problem odtwarzania sygnału rzadkiego sprowadza się do rozwiązania nieoznaczonego układu liniowego. Posiadając niepełną liczbą m pomiarów i~macierzą pomiarową o~rozmiarze mxn, należy znaleźć to rozwiązanie równania (należące do nieskończonej rodziny rozwiązań, posiadających m-n stopni swobody), które najlepiej odtwarza sygnał wejściowy. Okazuje się, że często rozwiązaniem bliskim optymalnemu jest sygnał najbardziej rzadki, tzn. posiadający największą liczbę zerowych współczynników. W~ostatnich latach przygotowano szereg metod, różniących się złożonością obliczeniową i~konieczną liczbą pomiarów. W~gruncie rzeczy wszystkie sprowadzają się jednak do rozwiązania równań Newtona.  [Tu opisać L2, L1, L0, Indyk – falki, MRI – Fourier].

Poniższy raport składa się z sześciu rozdziałów. Po wstępie następuje rozdział poświęcony złożoności czasowej i pamięciowej postawionego problemu. Rozdział trzeci przedstawia koncepcję jego rozwiązania, a czwarty - procedury testowe. Rozdziały 5 i 6 opisują odpowiednio wyniki przeprowadzonych eksperymentów oraz podsumowanie. Szczegółowy opis zadania znajduje się w załączniku.
\subsection{Cele i założenia projektu}
Głównym celem projektu była implementacja wybranego algorytmu odtwarzania sygnału rzadkiego, wykorzystując do obliczeń kartę graficzną. Zadaniem dodatkowym była symulacja jednopikselowej kamery, wykorzystując kamerę firmy Jai, dostępną w sali laboratoryjnej. Projekt miał umożliwić znaczne przyspieszenie istniejących rozwiązań, opartych głównie o skrypty środowiska Matlab/Simulink, a także umożliwić odtwarzanie obrazów o większej skali (docelowo obraz w pełnej rozdzielczości kamery, czyli 2560 x 2048 pikseli). Z uwagi na przyjęty sposób generowania pomiarów oraz przede wszystkim - konieczność przechowywania macierzy pomiarowej w pamięci, okazało się to kompletnie nierealne.
\subsection{Zarys proponowanego rozwiązania}
Projekt rozpoczęto od rozległych badań literaturowych. Oprócz zrozumienia zagadnienia, próbowano rozeznać się w dostępnych metodach, a w szczególności – ich potencjalnej wydajności mplementuje algorytm TQVC[Link]. Jest on atrakcyjny ze względu na stosunkową łatwość w implementacji, jak i wysoką skuteczność. Funkcją celu, minimalizowaną w tej metodzie jest  totalna wariacja (ang. \textbf{TV} – \textit{Total Varation}). Jej istota polega na założeniu, że \textit{gradient} zdjęcia jest rzadki.