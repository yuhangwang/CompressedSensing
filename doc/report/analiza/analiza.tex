\section{Analiza złożoności i estymacja zapotrzebowania na zasoby}
Pierwszym i niezbędnym krokiem przy pracy na stanowisku laboratoryjnym było zapoznanie się z dostępną architekturą sprzętową oraz wybór środowiska programistycznego. Następnie, zapoznano się i opisano standard OpenCL, w ramach którego implementowano algorytm rekonstrukcji obrazu na kartach graficznych. Należało bowiem poznać szczególne wymagania oraz ograniczenia wynikające z jego użycia. Dalszą część rozdziału stanowi oszacowanie złożoności pamięciowej i obliczeniowej wybranego algorytmu odtwarzania obrazu.
\subsection{Sprzęt uruchomieniowy}
Dostępny w laboratorium komputer składał się z wielordzeniowego procesora klasy Intel i5 oraz dwóch kart graficznych firmy Nvidia GeForce GTX670. W poniższej tabeli zebrano ich główne parametry:

\begin{table}[H]
\begin{center}
\begin{tabular}{|c|c|}
\hline
         & Nvidia GeForce GTX 670  \\
\hline
        Liczba rdzeni & 1344 \\
\hline
        Częstotliwość bazowa & 915 [MHz] \\
\hline
        Pojemność pamięci & 2048 [MB] \\
\hline
        Magistrala & 256 [bit] \\
\hline
\end{tabular} 
\caption{Wybrane parametry kart graficznych Nvidia GTX 670}
\label{tab:PGU}
\end{center}
\end{table}

Liczba rdzeni wykorzystywanej przez nas karty graficznej jest wprost imponująca. Potencjalnie, wykorzystanie tej ilości jednostek obliczeniowych może dać kolosalne korzyści czasowe. Program, który w skuteczny sposób wykorzystuje powyższe zasoby, musi dokonać kilku niskopoziomowych optymalizacji. Doskonałą lekturę w tym zakresie stanowiła książka opisująca bibliotekę OpenCL \cite{Scarpino2012}. Po pierwsze, należy minimalizować liczbę rozgałęzień algorytmu. Po drugie, zadania na karcie graficznej należy dzielić tak, by zależności między wątkami były jak najmniejsze - konieczność synchronizacji znacznie zmniejsza wydajność programu. Po trzecie, należy minimalizować liczbę odwołań wątków karty graficznej do pamięci globalnej - podręczne zasoby lokalne charakteryzują się znacznie mniejszym opóźnieniem. W końcu, należy liczyć się z każdorazowym opóźnieniem obliczeń na GPU, ze względu na dwukrotny transfer danych między hostem a urządzeniem.

Zdecydowano się na implementację programu w języku C++, z wykorzystaniem środowiska Visual Studio. Wynikało to z kilku przyczyn. Po pierwsze, wybór był konsekwencją doświadczenia zawodowego autora. Pisanie kodu w znanym sobie języku i środowisku znacznie przyspieszyło pracę. Po drugie, przygotowany SDK kamery Jai zawierał binding w tym właśnie języku, co znacznie uprościło integrację tego modułu programu. C++ pozwolił twórcom z jednej strony zdekomponować problem obiektowo, a z drugiej - skompilować program natywnie, licząc na optymalną wydajność. Z perspektywy czasu trzeba jednak rozważyć, czy nie lepiej wykorzystać Pythona, z uwagi na duże wsparcie biblioteczne tego języka, a także czytelność i zwartość kodu. Ostatecznym celem programu było bowiem maksymalne wykorzystanie karty graficznej, z minimalną ingerencją komputera-hosta. Ponadto, krytyczne dla wydajności procedury można by zawrzeć w formie skompilowanej biblioteki języka C. Ciekawym byłoby zatem porównać nakład pracy przygotowanych programów i ich wydajność.

\subsection{Standard OpenCL}
Jednym z głównym wyzwań projektu było zaznajomienie się i wykorzystanie standardu OpenCL \cite{OpenCL} do programowania w heterogenicznym środowisku obliczeniowym. W przeciwieństwie do rozwijanego przez firmę Nvidia standardu CUDA, OpenCL może być wykorzystywany na różnych platformach sprzętowych, o ile tylko standard jest wspierany przez producenta. Choć OpenCL definiowuje tylko interfejs programistyczny w języku C oraz wrapper w C++, dostępne są również jego realizacje w Javie, Pythonie i innych. W chwili pisania raportu, opublikowano wersję 2.0 tego dokumentu. Niestety, żaden z producentów kart graficznych jeszcze go nie wspiera. Posługiwano się więc interfejsem programistycznym standardu w wersji 1.2.

Na podstawowe pojęcia w OpenCL składają się: \textit{host} i \textit{device}, na których wykonywana jest aplikacja. \textit{Host} to zwykle CPU, wysyłające rozkazy do GPU (\textit{device}), w szczególności w celu wykonywania \textit{kerneli}, czyli dynamicznie kompilowanych programów. Pojedynczy rdzeń procesora karty graficznej zwany jest \textit{work-itemem}. Pewien zbiór tychże stanowi \textit{work-group}. Jednostki obliczeniowe rozróżniane są globalnym identyfikatorem \textit{globalID}, a grupy - \textit{localID}. 
Pamięć \textit{device} możemy podzielić na 4 typy: private, local, constant, global. Ich dostęp możliwy jest odpowiednio: \textit{work-item}, \textit{work-group} oraz cały \textit{device}. Cechują się one kolejno rosnącą pojemnością. Niestety, wielkość pamięci idzie w parze z coraz dłuższym czasem dostępu.

\subsection{Wykorzystanie biblioteki ViennaCL w projekcie}
Pokrótce analizując kod źródłowy pakietu $\ell_1$-MAGIC, szybko zrozumiano, że napisanie i przetestowanie \textit{kerneli} dla karty graficznej będzie zadaniem karkołomnym. Standard w swej naturze jest bardzo niskopoziomowy, co z jednej strony zapewnia elastyczność - ale z drugiej, niską efektywność pracy. Zdolność do debugowania kodu przeznaczonego na kartę graficzną jest ograniczona. W dodatku, współbieżna praca wielu wątków programu jest częstym błędem występowania trudnych w zreprodukowaniu i zrozumieniu błędów. Z tego powodu zdecydowano się wykorzystać bibliotekę \textbf{ViennaCL} \cite{ViennaCL}, która oferuje wysokopoziomowy interfejs programistyczny do przygotowania i uruchamiania zadań na karcie graficznej, zawiera wiele przydatnych, gotowych funkcji, takich jak solvery układów liniowych, metody dekompozycji macierzy, a w końcu - jest niezwykle elastyczna, pozwalając na uruchamianie własnoręcznie napisanych \textit{kerneli}.

\subsection{Oszacowanie złożoności algorytmu TVQC}
